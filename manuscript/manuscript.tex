% Options for packages loaded elsewhere
\PassOptionsToPackage{unicode}{hyperref}
\PassOptionsToPackage{hyphens}{url}
\PassOptionsToPackage{dvipsnames,svgnames,x11names}{xcolor}
%
\documentclass[
]{agujournal2019}

\usepackage{amsmath,amssymb}
\usepackage{iftex}
\ifPDFTeX
  \usepackage[T1]{fontenc}
  \usepackage[utf8]{inputenc}
  \usepackage{textcomp} % provide euro and other symbols
\else % if luatex or xetex
  \usepackage{unicode-math}
  \defaultfontfeatures{Scale=MatchLowercase}
  \defaultfontfeatures[\rmfamily]{Ligatures=TeX,Scale=1}
\fi
\usepackage{lmodern}
\ifPDFTeX\else  
    % xetex/luatex font selection
\fi
% Use upquote if available, for straight quotes in verbatim environments
\IfFileExists{upquote.sty}{\usepackage{upquote}}{}
\IfFileExists{microtype.sty}{% use microtype if available
  \usepackage[]{microtype}
  \UseMicrotypeSet[protrusion]{basicmath} % disable protrusion for tt fonts
}{}
\makeatletter
\@ifundefined{KOMAClassName}{% if non-KOMA class
  \IfFileExists{parskip.sty}{%
    \usepackage{parskip}
  }{% else
    \setlength{\parindent}{0pt}
    \setlength{\parskip}{6pt plus 2pt minus 1pt}}
}{% if KOMA class
  \KOMAoptions{parskip=half}}
\makeatother
\usepackage{xcolor}
\setlength{\emergencystretch}{3em} % prevent overfull lines
\setcounter{secnumdepth}{5}
% Make \paragraph and \subparagraph free-standing
\makeatletter
\ifx\paragraph\undefined\else
  \let\oldparagraph\paragraph
  \renewcommand{\paragraph}{
    \@ifstar
      \xxxParagraphStar
      \xxxParagraphNoStar
  }
  \newcommand{\xxxParagraphStar}[1]{\oldparagraph*{#1}\mbox{}}
  \newcommand{\xxxParagraphNoStar}[1]{\oldparagraph{#1}\mbox{}}
\fi
\ifx\subparagraph\undefined\else
  \let\oldsubparagraph\subparagraph
  \renewcommand{\subparagraph}{
    \@ifstar
      \xxxSubParagraphStar
      \xxxSubParagraphNoStar
  }
  \newcommand{\xxxSubParagraphStar}[1]{\oldsubparagraph*{#1}\mbox{}}
  \newcommand{\xxxSubParagraphNoStar}[1]{\oldsubparagraph{#1}\mbox{}}
\fi
\makeatother


\providecommand{\tightlist}{%
  \setlength{\itemsep}{0pt}\setlength{\parskip}{0pt}}\usepackage{longtable,booktabs,array}
\usepackage{calc} % for calculating minipage widths
% Correct order of tables after \paragraph or \subparagraph
\usepackage{etoolbox}
\makeatletter
\patchcmd\longtable{\par}{\if@noskipsec\mbox{}\fi\par}{}{}
\makeatother
% Allow footnotes in longtable head/foot
\IfFileExists{footnotehyper.sty}{\usepackage{footnotehyper}}{\usepackage{footnote}}
\makesavenoteenv{longtable}
\usepackage{graphicx}
\makeatletter
\newsavebox\pandoc@box
\newcommand*\pandocbounded[1]{% scales image to fit in text height/width
  \sbox\pandoc@box{#1}%
  \Gscale@div\@tempa{\textheight}{\dimexpr\ht\pandoc@box+\dp\pandoc@box\relax}%
  \Gscale@div\@tempb{\linewidth}{\wd\pandoc@box}%
  \ifdim\@tempb\p@<\@tempa\p@\let\@tempa\@tempb\fi% select the smaller of both
  \ifdim\@tempa\p@<\p@\scalebox{\@tempa}{\usebox\pandoc@box}%
  \else\usebox{\pandoc@box}%
  \fi%
}
% Set default figure placement to htbp
\def\fps@figure{htbp}
\makeatother
% definitions for citeproc citations
\NewDocumentCommand\citeproctext{}{}
\NewDocumentCommand\citeproc{mm}{%
  \begingroup\def\citeproctext{#2}\cite{#1}\endgroup}
\makeatletter
 % allow citations to break across lines
 \let\@cite@ofmt\@firstofone
 % avoid brackets around text for \cite:
 \def\@biblabel#1{}
 \def\@cite#1#2{{#1\if@tempswa , #2\fi}}
\makeatother
\newlength{\cslhangindent}
\setlength{\cslhangindent}{1.5em}
\newlength{\csllabelwidth}
\setlength{\csllabelwidth}{3em}
\newenvironment{CSLReferences}[2] % #1 hanging-indent, #2 entry-spacing
 {\begin{list}{}{%
  \setlength{\itemindent}{0pt}
  \setlength{\leftmargin}{0pt}
  \setlength{\parsep}{0pt}
  % turn on hanging indent if param 1 is 1
  \ifodd #1
   \setlength{\leftmargin}{\cslhangindent}
   \setlength{\itemindent}{-1\cslhangindent}
  \fi
  % set entry spacing
  \setlength{\itemsep}{#2\baselineskip}}}
 {\end{list}}
\usepackage{calc}
\newcommand{\CSLBlock}[1]{\hfill\break\parbox[t]{\linewidth}{\strut\ignorespaces#1\strut}}
\newcommand{\CSLLeftMargin}[1]{\parbox[t]{\csllabelwidth}{\strut#1\strut}}
\newcommand{\CSLRightInline}[1]{\parbox[t]{\linewidth - \csllabelwidth}{\strut#1\strut}}
\newcommand{\CSLIndent}[1]{\hspace{\cslhangindent}#1}

\usepackage{url} %this package should fix any errors with URLs in refs.
\usepackage{lineno}
\usepackage[inline]{trackchanges} %for better track changes. finalnew option will compile document with changes incorporated.
\usepackage{soul}
\linenumbers
\makeatletter
\@ifpackageloaded{caption}{}{\usepackage{caption}}
\AtBeginDocument{%
\ifdefined\contentsname
  \renewcommand*\contentsname{Table of contents}
\else
  \newcommand\contentsname{Table of contents}
\fi
\ifdefined\listfigurename
  \renewcommand*\listfigurename{List of Figures}
\else
  \newcommand\listfigurename{List of Figures}
\fi
\ifdefined\listtablename
  \renewcommand*\listtablename{List of Tables}
\else
  \newcommand\listtablename{List of Tables}
\fi
\ifdefined\figurename
  \renewcommand*\figurename{Figure}
\else
  \newcommand\figurename{Figure}
\fi
\ifdefined\tablename
  \renewcommand*\tablename{Table}
\else
  \newcommand\tablename{Table}
\fi
}
\@ifpackageloaded{float}{}{\usepackage{float}}
\floatstyle{ruled}
\@ifundefined{c@chapter}{\newfloat{codelisting}{h}{lop}}{\newfloat{codelisting}{h}{lop}[chapter]}
\floatname{codelisting}{Listing}
\newcommand*\listoflistings{\listof{codelisting}{List of Listings}}
\makeatother
\makeatletter
\makeatother
\makeatletter
\@ifpackageloaded{caption}{}{\usepackage{caption}}
\@ifpackageloaded{subcaption}{}{\usepackage{subcaption}}
\makeatother

\usepackage{bookmark}

\IfFileExists{xurl.sty}{\usepackage{xurl}}{} % add URL line breaks if available
\urlstyle{same} % disable monospaced font for URLs
\hypersetup{
  pdftitle={Shifting paradigms in Ocean Color: Bayesian Inference for Uncertainty-Aware Chlorophyll Estimation},
  pdfauthor={Erdem M. Karaköylü; Susanne E. Craig},
  colorlinks=true,
  linkcolor={blue},
  filecolor={Maroon},
  citecolor={Blue},
  urlcolor={Blue},
  pdfcreator={LaTeX via pandoc}}


\journalname{Geophysical Research Letters}

\draftfalse

\begin{document}
\title{Shifting paradigms in Ocean Color: Bayesian Inference for
Uncertainty-Aware Chlorophyll Estimation}

\authors{Erdem M. Karaköylü\affil{1}, Susanne E. Craig\affil{2}}
\affiliation{1}{Consultant, }\affiliation{2}{NASA, }
\correspondingauthor{Erdem M. Karaköylü}{erdemk@protonmail.com}


\begin{abstract}
Placeholder
\end{abstract}

\section*{Plain Language Summary}
Placeholder




\subsection{Introduction}\label{introduction}

\subsubsection{Historical Context of Chlorophyll
Algorithms}\label{historical-context-of-chlorophyll-algorithms}

Satellite ocean color observations have long been fundamental for
monitoring marine ecosystems, as they enable the global estimation of
chlorophyll‑a (\(Chl_a\)) ---a key indicator of phytoplankton biomass
and ocean productivity. Early empirical algorithms, notably developed by
O'Reilly et al. (John E. O'Reilly et al., 1998; John E. O'Reilly et al.,
2000), established the \(OCx\) family (where \(x\) denotes the number of
bands used) of polynomial regression models. These models relate
blue-to-green reflectance ratios (after log‑transformation) to in situ
\(Chl_a\), employing either straight band ratios (BR) or maximum band
ratios (MBR)---the latter selecting the highest available blue-to-green
ratio for any given observation as input to a high‑order polynomial.
These formulations have served as the operational foundation for
chlorophyll‑a products across a broad range of satellite ocean color
sensors---from the pioneering Coastal Zone Color Scanner (CZCS) through
SeaWiFS, MODIS, and MERIS to more recent missions---offering a
straightforward and robust approach for Case‑1 waters. However, their
performance is more limited in optically complex Case‑2 waters and
remains sensitive to atmospheric correction errors.

Subsequent refinements were introduced to address these deficiencies.
For example, Hu et al. (Hu et al., 2012) proposed a Color Index (CI)
formulation that employs a band‑difference approach to reduce
sensitivity to residual atmospheric errors and instrument noise, with
further improvements enhancing inter‑sensor agreement (Hu et al., 2019).
The increasing availability of calibration data (e.g.,
(\textbf{Valente2015?})) and ongoing algorithmic improvements have led
to the development of additional variants of the \(OCx\)
algorithms---specifically, the OC5 and OC6 formulations. O'Reilly and
Werdell (John E. O'Reilly \& Werdell, 2019) maintain that OC5 extends
the spectral basis by incorporating the 412\,nm band, thereby exploiting
its strong signal in clear, oligotrophic waters, while OC6 replaces the
traditional denominator with the mean of the 555 and 670\,nm
reflectances, with the aim of improving the dynamic range at low
chlorophyll concentrations. In total, (John E. O'Reilly \& Werdell,
2019) propose 65 versions of BR/MBR \(OCx\) type algorithms for 25
sensors---on average, two or more variants per sensor. With this
arsenal, it is hoped, researchers are better equipped to address the
wide array of bio‑optical environments encountered in global ocean color
applications.

\subsubsection{Limitations of Existing
Approaches}\label{limitations-of-existing-approaches}

Regrettably, the development of traditional ocean color predictive
models relies on a fundamental statistical error that plagues most of
observational science today; that of conflating sampling probability,
with inferential probability. Consider data \(D\) that include
predictive features (e.g.~Remote sensing reflectance - Rrs) and
prediction targets (e.g.~\(Chl_a\) or phytoplankton absorption), and a
model \(M\) (e.g.~OCx) hypothesized to adequately represent the
statistical association between the two. The sampling probability
\(p(D|M)\) also referred to as the likelihood, is the probability of
\(D\) conditioned on \(M\) being ``true''. The common model fitting
practice is to maximize \(p(D|M)\) - likelihood maximization - achieved
by tuning \(M\)'s parameters. The unspoken assumption is that making
\(M\) fit \(D\) as well as possible yields the best version of \(M\)
given \(D\) i.e.~maximizing \(p(M|D)\), the inferential probability.
This sneaky swapping from \(p(D|M)\) to \(p(M|D)\) is a violation of the
rules of conditional probability, Bayes' theorem in particular.
Jaynes(\textbf{jaynes2003?}) and later Clayton through his highly
readable account (\textbf{clayton?}) are some of the more notable
whistleblowers. Occasionally, in simple or well-behaved problems, with a
large amount of data and no prior information available the maxima of
both \(p(M|D)\) and \(p(D|M)\) can indeed converge on the same answer.
Yet there is no guarantee predictability for this outcome however, and
the fallout of this abuse is non-negligible, most noticeably in terms of
models that don't generalize well, ad-hoc uncertainty estimation
attempts, and more broadly a replication crisis.

Some researchers have attempted to deal with these issues. (Seegers et
al., 2018) e.g.~have proposed alternative metrics to circumvent the
inadequate assumptions of the frequentist approach. Others have tried to
go a step futher and switch to the Bayesian paradigm. E.g. (Frouin \&
Pelletier, 2013) have proposed a Bayesian inversion scheme for
atmospheric correction. (\textbf{shi2015?}) proposed a probabilistic
method to merge data from different sensors. (Craig \& Karaköylü, 2019)
have proposed a Bayesian neural network approach using Hamiltonian Monte
Carlo sampling to retrieve Inherent Optical Properties (IOP) from
Top-of-the-atmosphere (TOA) radiance. (\textbf{werther2022?}) used
Monte-Carlo dropout as deep neural net training method to obtain
prediction uncertainty. (Erickson et al., 2023) have proposed using
conjugate Gaussian prior and likelihood to predict IOPs using GIOP as a
forward model. (\textbf{hammout2024?}) have proposed a Bayesian Neural
Network trained using Stochastic Variational Inference to predict
\(Chl_a\) from ocean color observations. Yet most of these approaches
retain variable levels of frequentism by applying only part of what is
commonly referred to as the Bayesian workflow(\textbf{betancourt2019?}).

In this paper we propose a modified workflow to Bayesian modeling for
Ocean Color emote sensing. We propose that this workflow include; (1)
one or more model building, (2) prior formulation via expert knowledge
elicitation and/or data-free simulation, (3) fitting model to data, (4)
model validation, comparison and selection, (5) interpretation of
results via posterior analysis, (6) prediction on new observation, (7)
update the posterior when new labeled data becomes available. As
illustration we recast some of the pre-existing \(OCx\) models into
their probabilistic version. To illustrate model comparison and
selection we also build and evaluate Bayesian Additive Tree Regregssion
(BART; (Chipman et al., 2010)) type model.

\subsection{Methods}\label{methods}

\subsubsection{Prior Elicitation and Model
Formulation}\label{prior-elicitation-and-model-formulation}

\subsubsection{Data Preprocessing}\label{data-preprocessing}

\subsubsection{}\label{section}

\section{Data Preprocessing}

Data for this study were acquired from multiple satellite ocean color
sensors and corresponding in situ chlorophyll-(a) measurements obtained
from sources such as the NASA Bio-Optical Marine Algorithm Data set
(NOMAD) and the compilation by Valente et al.~(2015). To ensure
consistency across sensors, the spectral reflectance data ((R\_\{rs\}))
were interpolated as needed to common wavelength centers.

For the empirical (OCx) formulation, blue-to-green band ratios were
computed for each observation. In particular, the maximum band ratio
(MBR) was determined by taking the highest value among the available
blue-band ratios (e.g., (Rrs(443)/Rrs(555)), (Rrs(490)/Rrs(555)), and
(Rrs(510)/Rrs(555))). This maximum value was then log-transformed: \[
\log R = \log_{10}\left(\frac{R_{rs}(\lambda_{\text{blue}})}{R_{rs}(555)}\right).
\]

For the Color Index (CI) formulation of Hu et al.~(2012), the CI was
calculated as: \[
\text{CI} = R_{rs}(555) - \left[\,R_{rs}(443) + \frac{555-443}{670-443}\Bigl(R_{rs}(670) - R_{rs}(443)\Bigr)\,\right],
\] and the corresponding in situ chlorophyll-(a) concentrations were
log-transformed: \[
\log \text{Chl} = \log_{10}(\text{Chl}).
\]

These transformations standardize the data to a common scale, ensuring
that variability is appropriately captured for subsequent regression and
uncertainty quantification. Detailed descriptions of the interpolation
methods and quality control procedures are provided in the Supplementary
Material.

\section{Statement of Contribution}\label{statement-of-contribution}

In this study, we develop and demonstrate a new global chlorophyll
retrieval model based on Bayesian Additive Regression Trees implemented
in PyMC. We train the BART model on a large, standardized dataset of
satellite remote-sensing reflectance (Rrs) spectra matched with in situ
chlorophyll measurements, using log₁₀-transformed Chl-a as the response
to stabilize variance. The resulting model is applied globally to
produce chlorophyll-a estimates from multi-spectral satellite data, with
associated uncertainty estimates for each prediction. We show that this
Bayesian tree-based model can serve as a general-purpose ocean color
algorithm that is sensor-agnostic (provided reflectances are harmonized
to common wavebands), interpretable, and uncertainty-aware. Unlike
conventional empirical algorithms, the BART approach allows users to
examine the inferred Rrs--Chl relationships and trust the model's
performance across regimes, while also quantifying confidence in each
retrieval. This work thus contributes a novel methodological advance to
satellite ocean color science: a unified chlorophyll retrieval model
that marries the strengths of empirical algorithms (global applicability
and simplicity) with the benefits of modern Bayesian machine learning
(flexibility, interpretability, and rigorous uncertainty
quantification). Our introduction of BART for global chlorophyll
prediction opens the door for more robust monitoring of ocean
biogeochemistry and improved integration of ocean color data into
scientific and management applications.

\section{Acknowledgments}\label{acknowledgments}

\section{Open research}\label{open-research}

\section*{References}\label{references}
\addcontentsline{toc}{section}{References}

\phantomsection\label{refs}
\begin{CSLReferences}{1}{0}
\vspace{1em}

\bibitem[\citeproctext]{ref-chipman2010bart}
Chipman, H. A., George, E. I., \& McCulloch, R. E. (2010). BART:
Bayesian additive regression trees. \emph{The Annals of Applied
Statistics}, \emph{4}(1), 266--298.
\url{https://doi.org/10.1214/09-AOAS285}

\bibitem[\citeproctext]{ref-craig2019}
Craig, S. E., \& Karaköylü, E. M. (2019). Bayesian models for deriving
biogeochemical information from satellite ocean color.
\emph{EarthArXiv}. \url{https://doi.org/10.31223/osf.io/shp6y}

\bibitem[\citeproctext]{ref-erickson2023}
Erickson, Z. K., McKinna, L. I. W., Werdell, P. J., \& Cetinić, I.
(2023). Bayesian approach to a generalized inherent optical property
model. \emph{Optics Express}, \emph{31}, 22790--22801.
\url{https://doi.org/10.1364/oe.486581}

\bibitem[\citeproctext]{ref-frouin2013}
Frouin, R., \& Pelletier, B. (2013). Bayesian methodology for ocean
color remote sensing. Retrieved from
\url{https://hal.archives-ouvertes.fr/hal-00822032}

\bibitem[\citeproctext]{ref-hu2012novel}
Hu, C., Lee, Z., \& Franz, B. (2012). Chlorophyll aalgorithms for
oligotrophic oceans: A novel approach based on three-band reflectance
difference. \emph{Journal of Geophysical Research: Oceans},
\emph{117}(C1).
https://doi.org/\url{https://doi.org/10.1029/2011JC007395}

\bibitem[\citeproctext]{ref-hu2019improving}
Hu, C., Feng, L., Lee, Z., Franz, B. A., Bailey, S. W., Werdell, P. J.,
\& Proctor, C. W. (2019). Improving satellite global chlorophyll a data
products through algorithm refinement and data recovery. \emph{Journal
of Geophysical Research: Oceans}, \emph{124}(3), 1524--1543.
https://doi.org/\url{https://doi.org/10.1029/2019JC014941}

\bibitem[\citeproctext]{ref-oreilly2019}
O'Reilly, John E., \& Werdell, P. J. (2019). Chlorophyll algorithms for
ocean color sensors - OC4, OC5 \& OC6. \emph{Remote Sensing of
Environment}, \emph{229}, 32--47.
https://doi.org/\url{https://doi.org/10.1016/j.rse.2019.04.021}

\bibitem[\citeproctext]{ref-oreilly1998}
O'Reilly, John E., Maritorena, S., Mitchell, B. G., Siegel, D. A.,
Carder, K. L., Garver, S. A., et al. (1998). Ocean color chlorophyll
algorithms for SeaWiFS. \emph{Journal of Geophysical Research: Oceans},
\emph{103}(C11), 24937--24953.
https://doi.org/\url{https://doi.org/10.1029/98JC02160}

\bibitem[\citeproctext]{ref-oreilly2000}
O'Reilly, John E., Maritorena, S., Siegel, D. A., O'Brien, M. C., Toole,
D., Mitchell, B. G., et al. (2000). Ocean color chlorophyll a algorithms
for SeaWiFS, OC2, and OC4: Version 4. \emph{SeaWiFS Postlaunch
Calibration and Validation Analyses, Part}, \emph{3}, 9--23.

\bibitem[\citeproctext]{ref-seegers2018}
Seegers, B. N., Stumpf, R. P., Schaeffer, B. A., Loftin, K. A., \&
Werdell, P. J. (2018). Performance metrics for the assessment of
satellite data products: An ocean color case study. \emph{Opt. Express},
\emph{26}(6), 7404--7422. \url{https://doi.org/10.1364/OE.26.007404}

\end{CSLReferences}




\end{document}
