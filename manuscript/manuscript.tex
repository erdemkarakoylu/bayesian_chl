% Options for packages loaded elsewhere
\PassOptionsToPackage{unicode}{hyperref}
\PassOptionsToPackage{hyphens}{url}
\PassOptionsToPackage{dvipsnames,svgnames,x11names}{xcolor}
%
\documentclass[
]{agujournal2019}

\usepackage{amsmath,amssymb}
\usepackage{iftex}
\ifPDFTeX
  \usepackage[T1]{fontenc}
  \usepackage[utf8]{inputenc}
  \usepackage{textcomp} % provide euro and other symbols
\else % if luatex or xetex
  \usepackage{unicode-math}
  \defaultfontfeatures{Scale=MatchLowercase}
  \defaultfontfeatures[\rmfamily]{Ligatures=TeX,Scale=1}
\fi
\usepackage{lmodern}
\ifPDFTeX\else  
    % xetex/luatex font selection
\fi
% Use upquote if available, for straight quotes in verbatim environments
\IfFileExists{upquote.sty}{\usepackage{upquote}}{}
\IfFileExists{microtype.sty}{% use microtype if available
  \usepackage[]{microtype}
  \UseMicrotypeSet[protrusion]{basicmath} % disable protrusion for tt fonts
}{}
\makeatletter
\@ifundefined{KOMAClassName}{% if non-KOMA class
  \IfFileExists{parskip.sty}{%
    \usepackage{parskip}
  }{% else
    \setlength{\parindent}{0pt}
    \setlength{\parskip}{6pt plus 2pt minus 1pt}}
}{% if KOMA class
  \KOMAoptions{parskip=half}}
\makeatother
\usepackage{xcolor}
\setlength{\emergencystretch}{3em} % prevent overfull lines
\setcounter{secnumdepth}{5}
% Make \paragraph and \subparagraph free-standing
\makeatletter
\ifx\paragraph\undefined\else
  \let\oldparagraph\paragraph
  \renewcommand{\paragraph}{
    \@ifstar
      \xxxParagraphStar
      \xxxParagraphNoStar
  }
  \newcommand{\xxxParagraphStar}[1]{\oldparagraph*{#1}\mbox{}}
  \newcommand{\xxxParagraphNoStar}[1]{\oldparagraph{#1}\mbox{}}
\fi
\ifx\subparagraph\undefined\else
  \let\oldsubparagraph\subparagraph
  \renewcommand{\subparagraph}{
    \@ifstar
      \xxxSubParagraphStar
      \xxxSubParagraphNoStar
  }
  \newcommand{\xxxSubParagraphStar}[1]{\oldsubparagraph*{#1}\mbox{}}
  \newcommand{\xxxSubParagraphNoStar}[1]{\oldsubparagraph{#1}\mbox{}}
\fi
\makeatother


\providecommand{\tightlist}{%
  \setlength{\itemsep}{0pt}\setlength{\parskip}{0pt}}\usepackage{longtable,booktabs,array}
\usepackage{calc} % for calculating minipage widths
% Correct order of tables after \paragraph or \subparagraph
\usepackage{etoolbox}
\makeatletter
\patchcmd\longtable{\par}{\if@noskipsec\mbox{}\fi\par}{}{}
\makeatother
% Allow footnotes in longtable head/foot
\IfFileExists{footnotehyper.sty}{\usepackage{footnotehyper}}{\usepackage{footnote}}
\makesavenoteenv{longtable}
\usepackage{graphicx}
\makeatletter
\newsavebox\pandoc@box
\newcommand*\pandocbounded[1]{% scales image to fit in text height/width
  \sbox\pandoc@box{#1}%
  \Gscale@div\@tempa{\textheight}{\dimexpr\ht\pandoc@box+\dp\pandoc@box\relax}%
  \Gscale@div\@tempb{\linewidth}{\wd\pandoc@box}%
  \ifdim\@tempb\p@<\@tempa\p@\let\@tempa\@tempb\fi% select the smaller of both
  \ifdim\@tempa\p@<\p@\scalebox{\@tempa}{\usebox\pandoc@box}%
  \else\usebox{\pandoc@box}%
  \fi%
}
% Set default figure placement to htbp
\def\fps@figure{htbp}
\makeatother
% definitions for citeproc citations
\NewDocumentCommand\citeproctext{}{}
\NewDocumentCommand\citeproc{mm}{%
  \begingroup\def\citeproctext{#2}\cite{#1}\endgroup}
\makeatletter
 % allow citations to break across lines
 \let\@cite@ofmt\@firstofone
 % avoid brackets around text for \cite:
 \def\@biblabel#1{}
 \def\@cite#1#2{{#1\if@tempswa , #2\fi}}
\makeatother
\newlength{\cslhangindent}
\setlength{\cslhangindent}{1.5em}
\newlength{\csllabelwidth}
\setlength{\csllabelwidth}{3em}
\newenvironment{CSLReferences}[2] % #1 hanging-indent, #2 entry-spacing
 {\begin{list}{}{%
  \setlength{\itemindent}{0pt}
  \setlength{\leftmargin}{0pt}
  \setlength{\parsep}{0pt}
  % turn on hanging indent if param 1 is 1
  \ifodd #1
   \setlength{\leftmargin}{\cslhangindent}
   \setlength{\itemindent}{-1\cslhangindent}
  \fi
  % set entry spacing
  \setlength{\itemsep}{#2\baselineskip}}}
 {\end{list}}
\usepackage{calc}
\newcommand{\CSLBlock}[1]{\hfill\break\parbox[t]{\linewidth}{\strut\ignorespaces#1\strut}}
\newcommand{\CSLLeftMargin}[1]{\parbox[t]{\csllabelwidth}{\strut#1\strut}}
\newcommand{\CSLRightInline}[1]{\parbox[t]{\linewidth - \csllabelwidth}{\strut#1\strut}}
\newcommand{\CSLIndent}[1]{\hspace{\cslhangindent}#1}

\usepackage{url} %this package should fix any errors with URLs in refs.
\usepackage{lineno}
\usepackage[inline]{trackchanges} %for better track changes. finalnew option will compile document with changes incorporated.
\usepackage{soul}
\linenumbers
\makeatletter
\@ifpackageloaded{caption}{}{\usepackage{caption}}
\AtBeginDocument{%
\ifdefined\contentsname
  \renewcommand*\contentsname{Table of contents}
\else
  \newcommand\contentsname{Table of contents}
\fi
\ifdefined\listfigurename
  \renewcommand*\listfigurename{List of Figures}
\else
  \newcommand\listfigurename{List of Figures}
\fi
\ifdefined\listtablename
  \renewcommand*\listtablename{List of Tables}
\else
  \newcommand\listtablename{List of Tables}
\fi
\ifdefined\figurename
  \renewcommand*\figurename{Figure}
\else
  \newcommand\figurename{Figure}
\fi
\ifdefined\tablename
  \renewcommand*\tablename{Table}
\else
  \newcommand\tablename{Table}
\fi
}
\@ifpackageloaded{float}{}{\usepackage{float}}
\floatstyle{ruled}
\@ifundefined{c@chapter}{\newfloat{codelisting}{h}{lop}}{\newfloat{codelisting}{h}{lop}[chapter]}
\floatname{codelisting}{Listing}
\newcommand*\listoflistings{\listof{codelisting}{List of Listings}}
\makeatother
\makeatletter
\makeatother
\makeatletter
\@ifpackageloaded{caption}{}{\usepackage{caption}}
\@ifpackageloaded{subcaption}{}{\usepackage{subcaption}}
\makeatother

\usepackage{bookmark}

\IfFileExists{xurl.sty}{\usepackage{xurl}}{} % add URL line breaks if available
\urlstyle{same} % disable monospaced font for URLs
\hypersetup{
  pdftitle={Shifting paradigms in Ocean Color: Bayesian Inference for Uncertainty-Aware Chlorophyll Estimation},
  pdfauthor={Erdem M. Karaköylü; Susanne E. Craig},
  colorlinks=true,
  linkcolor={blue},
  filecolor={Maroon},
  citecolor={Blue},
  urlcolor={Blue},
  pdfcreator={LaTeX via pandoc}}


\journalname{Geophysical Research Letters}

\draftfalse

\begin{document}
\title{Shifting paradigms in Ocean Color: Bayesian Inference for
Uncertainty-Aware Chlorophyll Estimation}

\authors{Erdem M. Karaköylü\affil{1}, Susanne E. Craig\affil{2}}
\affiliation{1}{Consultant, }\affiliation{2}{NASA, }
\correspondingauthor{Erdem M. Karaköylü}{erdemk@protonmail.com}


\begin{abstract}
Placeholder
\end{abstract}

\section*{Plain Language Summary}
Placeholder




\subsection{Introduction}\label{introduction}

\subsubsection{Historical Context of Chlorophyll
Algorithms}\label{historical-context-of-chlorophyll-algorithms}

Satellite ocean color observations have long been fundamental for
monitoring marine ecosystems, as they enable the global estimation of
chlorophyll‑a---a key indicator of phytoplankton biomass and ocean
productivity. Early empirical algorithms, notably developed by O'Reilly
et al. (John E. O'Reilly et al., 1998; John E. O'Reilly et al., 2000),
established the \(OCx\) family (where \(x\) denotes the number of bands
used) of polynomial regression models. These models relate blue-to-green
reflectance ratios (after log‑transformation) to in situ chlorophyll‑a,
employing either straight band ratios (BR) or maximum band ratios
(MBR)---the latter selecting the highest available blue-to-green ratio
for any given observation as input to a high‑order polynomial. These
formulations have served as the operational foundation for chlorophyll‑a
products across a broad range of satellite ocean color sensors---from
the pioneering Coastal Zone Color Scanner (CZCS) through SeaWiFS, MODIS,
and MERIS to more recent missions---offering a straightforward and
robust approach for Case‑1 waters. However, their performance is more
limited in optically complex Case‑2 waters and remains sensitive to
atmospheric correction errors.

Subsequent refinements were introduced to address these deficiencies.
For example, Hu et al. (Hu et al., 2012) proposed a Color Index (CI)
formulation that employs a band‑difference approach to reduce
sensitivity to residual atmospheric errors and instrument noise, with
further improvements enhancing inter‑sensor agreement (Hu et al., 2019).
The increasing availability of calibration data (e.g.,
(\textbf{Valente2015?})) and ongoing algorithmic improvements have led
to the development of additional variants of the \(OCx\)
algorithms---specifically, the OC5 and OC6 formulations. O'Reilly and
Werdell (John E. O'Reilly \& Werdell, 2019) maintain that OC5 extends
the spectral basis by incorporating the 412\,nm band, thereby exploiting
its strong signal in clear, oligotrophic waters, while OC6 replaces the
traditional denominator with the mean of the 555 and 670\,nm
reflectances, improving the dynamic range at low chlorophyll
concentrations. In total, (John E. O'Reilly \& Werdell, 2019) propose 65
versions of BR/MBR \(OCx\) type algorithms for 25 sensors---on average,
two or more variants per sensor. With this arsenal, it is hoped,
researchers are better equipped to address the wide array of bio‑optical
environments encountered in global ocean color applications.

\subsubsection{Limitations of Existing
Approaches}\label{limitations-of-existing-approaches}

Unfortunately, the development process for traditional ocean color
algorithms is founded on a statistical fallacy. Frequentism---the
classical approach---assumes that observations are random samples drawn
from the studied phenomena. These samples' probabilities are considered
the observed frequencies that capture the the true - albeit noised up
and potentially biased - behavior of the system. Under this framework,
if a phenomenon is hypothesized to be represented by a model \(M\) and
data \(D\) is collected, then the model's representativeness is
evaluated solely through the likelihood \(p(D∣M)\), which measures the
probability of observing the data given that \(M\) is true. The fallacy
arises from the mistaken belief that, with sufficiently large amounts of
data, one can simply ``flip'' the likelihood to obtain \(p(M∣D)\),
thereby yielding a probabilistic statement about \(M\). Compounding this
error is the imposition of numerous unverified assumptions---disguised
as objectivity---that lead frequentists to set the parameters of \(M\)
by maximizing \(p(D∣M)\), under the false premise that this provides the
best possible representation of the model. In essence, this fallacy
leads to a confusion between the sampling probability \(p(D∣M)\) and the
inferential probability \(p(M∣D)\); the latter when correctly obtained
quantifies the plausibility of the hypothesis represented by \(M\) -
precisely what we seek. The historical dimensions of this frequentist
illusion are fascinating in their own right, and we invite interested
readers to consult works such as McGrayne (\textbf{mcgrayne?}) and
Clayton (\textbf{clayton?}).

Bayesianism, Frequentism's archrival, provides the framoework to
establish the plausibility of \(M\) using observations through the
reinterpretation of how probability should be construed. As elaborated
extensively by Jaynes (\textbf{jaynes2003probability?}), and more
recently by practitioners like Gelman (\textbf{gelman2013bayesian?}),
and McElreath (\textbf{mcelreath2015statistical?}), probability is best
understood as a statement of our knowledge rather than as an inherent
property of the data. the Bayesian perspective treats the data, once
observed, as fixed and the model parameters as latent random variables
the uncertainty of which can---and should---be explicitly modeled via
prior distributions. Prior distributions are probabilistic constructs
that help workers encode prior knowledge. Priors can represents
knowledge elicited from subject matter experts, assumptions held by the
modeler or a complete absence thereof.

This conceptual framework dissolves the false dichotomy between
empirical and semi‑analytical approaches. Both are, in essence,
different hypotheses or approximations about how ocean color data are
generated; they merely provide alternative expressions of the likelihood
\(p(Chl | Rrs)\) to be updated in light of new data. By adopting a
Bayesian paradigm, our approach integrates prior knowledge with observed
data to yield a full posterior predictive distribution---offering not
only improved predictive performance but also principled uncertainty
quantification. In doing so, it circumvents the need for the
proliferation of narrowly tuned, sensor-specific variants (65 versions
for 25 sensors, as documented by O'Reilly and Werdell
(\textbf{OREILLY2019?})) that are symptomatic of ad hoc calibration
processes. Detailed methodological comparisons and additional model
refinements are provided in the Supplementary Material.

\subsubsection{Motivation for Bayesian Tree-Based
Models}\label{motivation-for-bayesian-tree-based-models}

To address these limitations, we explore Bayesian tree-based modeling --
specifically, Bayesian Additive Regression Trees (BART) -- as a novel
approach for ocean color chlorophyll retrieval. BART offers several
advantages for this application:

Direct Bayesian inference: As a Bayesian ensemble method, BART directly
targets the posterior distribution \(p(Chl | Rrs)\) rather than a purely
empirical fit. The model inherently provides a probabilistic mapping
from reflectance to chlorophyll. Rather than producing a single best-fit
value, BART returns a posterior mean and credible interval for each
prediction, naturally quantifying predictive uncertainty.

Uncertainty quantification: The BART framework includes uncertainty in
its predictions by design. Every estimated chlorophyll value comes with
a confidence range (e.g., 95\% credible interval) reflecting model
uncertainty and observational noise. This is a key improvement over
deterministic algorithms, enabling more reliable use of the data for
scientific and operational purposes where understanding uncertainty is
crucial e.g.~for forecasting trends.

Interpretability: Being a sum-of-trees model, BART can be interrogated
to understand the influence of each input variable. Tools such as
partial dependence plots allow us to visualize the relationship between
each spectral band and the predicted chlorophyll, marginalizing over
other bands. In contrast to high-order polynomial coefficients, these
tree-based partial dependencies are intuitively interpretable. For
instance, one can observe how changes in a particular waveband's
reflectance (holding others average) drive the chlorophyll estimate,
revealing any thresholds or non-linear responses.

BART's modeling approach is non-parametric and highly flexible, which
means it can capture complex, non-linear relationships between
multi-spectral Rrs inputs and chlorophyll without imposing a rigid
functional form. At the same time, its Bayesian regularization -- which
shrinks the contribution of each individual tree -- helps prevent
overfitting even with a rich model structure. This balance between
flexibility and regularization is well-suited to global ocean color
data, where the true Rrs--Chl relationship is known to be non-linear and
context-dependent, yet a model must generalize across diverse water
types.

The interpretability of the BART model is illustrated by partial
dependence analysis on the six input Rrs bands (centered approximately
at 412, 443, 490, 510, 555, and 670 nm). The partial dependence plots
reveal nuanced spectral--chlorophyll relationships and clear transitions
in the dominant wavelength influence across different chlorophyll
regimes. For example, the shorter blue wavelengths (412--443 nm) show a
strong inverse influence on predicted \(log(Chl_a)\) (higher blue
reflectance corresponds to lower chlorophyll) that gradually saturates
in the most oligotrophic waters, consistent with the leveling off of
blue-to-green ratios at very low Chl. The green band (\textasciitilde555
nm) exhibits a more complex, non-monotonic effect: at low reflectance
(clear water) its influence on the Chl estimate is minimal, but it
becomes increasingly important through intermediate reflectance ranges
-- reflecting the band-difference signal exploited by the CI algorithm
-- and then diminishes for very high chlorophyll where green reflectance
tends to flatten. In contrast, the red band (\textasciitilde670 nm) has
virtually no effect on the model's predictions until a threshold is
reached at elevated chlorophyll concentrations, after which its
influence steeply increases. This behavior aligns with optical
expectations, since red reflectance is negligible in low-Chl waters but
rises sharply once phytoplankton absorption in the blue saturates and
biomass is high. These partial dependence results indicate that the BART
model automatically learns the piecewise spectral logic that
oceanographers often handle via separate algorithms (blue/green ratios
for low-to-moderate Chl, red bands for high Chl). The ability to capture
such regime-dependent responses in a single unified model, and to
visualize them, underscores the interpretability and scientific insight
provided by the BART approach.

Statement of Contribution In this study, we develop and demonstrate a
new global chlorophyll retrieval model based on Bayesian Additive
Regression Trees implemented in PyMC. We train the BART model on a
large, standardized dataset of satellite remote-sensing reflectance
(Rrs) spectra matched with in situ chlorophyll measurements, using
log₁₀-transformed Chl-a as the response to stabilize variance. The
resulting model is applied globally to produce chlorophyll-a estimates
from multi-spectral satellite data, with associated uncertainty
estimates for each prediction. We show that this Bayesian tree-based
model can serve as a general-purpose ocean color algorithm that is
sensor-agnostic (provided reflectances are harmonized to common
wavebands), interpretable, and uncertainty-aware. Unlike conventional
empirical algorithms, the BART approach allows users to examine the
inferred Rrs--Chl relationships and trust the model's performance across
regimes, while also quantifying confidence in each retrieval. This work
thus contributes a novel methodological advance to satellite ocean color
science: a unified chlorophyll retrieval model that marries the
strengths of empirical algorithms (global applicability and simplicity)
with the benefits of modern Bayesian machine learning (flexibility,
interpretability, and rigorous uncertainty quantification). Our
introduction of BART for global chlorophyll prediction opens the door
for more robust monitoring of ocean biogeochemistry and improved
integration of ocean color data into scientific and management
applications.

\section{Acknowledgments}\label{acknowledgments}

\section{Open research}\label{open-research}

\section*{References}\label{references}
\addcontentsline{toc}{section}{References}

\phantomsection\label{refs}
\begin{CSLReferences}{1}{0}
\vspace{1em}

\bibitem[\citeproctext]{ref-hu2012novel}
Hu, C., Lee, Z., \& Franz, B. (2012). Chlorophyll aalgorithms for
oligotrophic oceans: A novel approach based on three-band reflectance
difference. \emph{Journal of Geophysical Research: Oceans},
\emph{117}(C1).
https://doi.org/\url{https://doi.org/10.1029/2011JC007395}

\bibitem[\citeproctext]{ref-hu2019improving}
Hu, C., Feng, L., Lee, Z., Franz, B. A., Bailey, S. W., Werdell, P. J.,
\& Proctor, C. W. (2019). Improving satellite global chlorophyll a data
products through algorithm refinement and data recovery. \emph{Journal
of Geophysical Research: Oceans}, \emph{124}(3), 1524--1543.
https://doi.org/\url{https://doi.org/10.1029/2019JC014941}

\bibitem[\citeproctext]{ref-oreilly2019}
O'Reilly, John E., \& Werdell, P. J. (2019). Chlorophyll algorithms for
ocean color sensors - OC4, OC5 \& OC6. \emph{Remote Sensing of
Environment}, \emph{229}, 32--47.
https://doi.org/\url{https://doi.org/10.1016/j.rse.2019.04.021}

\bibitem[\citeproctext]{ref-oreilly1998}
O'Reilly, John E., Maritorena, S., Mitchell, B. G., Siegel, D. A.,
Carder, K. L., Garver, S. A., et al. (1998). Ocean color chlorophyll
algorithms for SeaWiFS. \emph{Journal of Geophysical Research: Oceans},
\emph{103}(C11), 24937--24953.
https://doi.org/\url{https://doi.org/10.1029/98JC02160}

\bibitem[\citeproctext]{ref-oreilly2000}
O'Reilly, John E., Maritorena, S., Siegel, D. A., O'Brien, M. C., Toole,
D., Mitchell, B. G., et al. (2000). Ocean color chlorophyll a algorithms
for SeaWiFS, OC2, and OC4: Version 4. \emph{SeaWiFS Postlaunch
Calibration and Validation Analyses, Part}, \emph{3}, 9--23.

\end{CSLReferences}




\end{document}
